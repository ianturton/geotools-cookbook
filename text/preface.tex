\chapter*{Preface}
\section*{About this book}
This book is a gentle introduction to the GeoTools library. It is aimed at developers who have a geographic problem that they want to solve in Java. 

\section*{Who is this book for?}
You are a Java developer who has written a number of complex programs already, you know how to set up your development environment whether it is an IDE or a command line and text editor. You are happy using Maven to manage your code (or at least willing to learn) and you have a interest in geographic problems (or your boss has). However, you are smart and you don't want to have to write a whole geographic library for yourself from scratch. You have no time or energy to look up the Shapefile white paper \citep{ESRI1997} to see how to read your data in, or work out the projection needed to make the map equal area. 

So to make your life easier you've picked an existing library but now you have to learn how that works. This book will help you solve common problems using the GeoTools library.

The joy of Java programming is that it doesn't matter if you use a Linux machine, a windows box or even a Mac. But beware many parts of the GeoTools library depend on the AWT and the Java Service Provider mechanism neither of which work on Android devices at the time of writing, so if you want to write Android applications you will need to find another library. That's not to say you can't take some of the code and techniques shown here on Android just that there is a fairly good chance that it will not work as is.
 
\section*{About the author}

\section*{A brief history of GeoTools}